% !TEX program = xelatex
\documentclass[11pt, letterpaper]{exam}

\usepackage{amsmath}
\usepackage{amssymb}
\usepackage{graphicx}

\graphicspath{ {./images/} }

\renewcommand{\thepartno}{\roman{partno}}
\renewcommand{\thesubpart}{(\roman{subpart})}
\renewcommand{\subpartlabel}{\thesubpart} % removes dot
\qformat{Exercise \thequestion: \hfill}

\newcommand{\tab}{\hspace{.5cm}}
\newcommand{\includeimage}{\noindent\includegraphics[width=\linewidth]}


\title{Artificial Intelligence \\ Practical 1 \\ Tutorial 4}
\author{Sudharshini Thangamurugan, 2577870 \\ Jeffrey Solomon, 2577386 \\ Ilya Senatorov, 2576798}
\date{May 14, 2019}

% \includeimage{FIGURE}

\begin{document}

  \maketitle

  \begin{questions}

    \question
    \begin{parts}
      \setcounter{partno}{2}
      \part The BFS algorithm found the goal first.
      \setcounter{partno}{4}
      \part
      In this example, expanding nodes nodes right first will reduce the number of expanstions.
      This will reach the goal after expanding 10 nodes.

    \end{parts}
    \vspace{1em}
    \question

      \begin{parts}
        \part
        \begin{align*}
          & ((a \lor b) \land \neg b) \to c \\
          \text{(step 2) \tab} & \equiv \lnot((a \lor b) \land \neg b) \lor c \\
          \text{(step 3) \tab} & \equiv ((\neg a \lor \neg b) \land (\neg b \lor \neg b)) \lor c \\
          \text{(step 4) \tab} & \equiv (((a \lor \neg b) \lor c) \land ((b \lor \neg b)) \lor c)
        \end{align*}

      \end{parts}

    \vspace{1em}

  \end{questions}

\end{document}
