% !TEX program = xelatex
\documentclass[11pt, letterpaper]{exam}

\usepackage{amsmath}
\usepackage{amssymb}
\usepackage{graphicx}

\graphicspath{ {./images/} }

\renewcommand{\thepartno}{\roman{partno}}
\renewcommand{\thesubpart}{(\roman{subpart})}
\renewcommand{\subpartlabel}{\thesubpart} % removes dot
\qformat{Exercise \thequestion: \hfill}

\newcommand{\tab}{\hspace{.5cm}}
\newcommand{\includeimage}{\noindent\includegraphics[width=\linewidth]}

\title{Artificial Intelligence \\ Assignment 1 \\ Tutorial 3}
\author{Sudharshini Thangamurugan, 2577870 \\ Jeffrey Solomon, 2577386 \\ Ilya Senatorov, 2576798}
\date{April 30, 2019}

\begin{document}

  \maketitle

  \begin{questions}

  \question

    \begin{parts}

      \part A Performance Measure is a third-party measure to decide if the agent has performed well.
      It evaluates the state of the environment. A utility function, on the other hand, is how an agent evaluates the
      state of the environment itself, and can be used to decide actions. Although it won't stop moving even after
      everything is complete, doing nothing is not an action this agent can take. If it had a 'do nothing' action, then
      it would be irrational for not using it.
      \

      \part A learning agent receives performance measure feedback, and adjusts it's actions/performance accordingly.
      The utility function changes based on the performance measure.

    \end{parts}

  \vspace{1em}

  \question

    \begin{parts}

      \part Our size of P is 4, the possible percept sets are: \\
        \tab [A, Clean] , [A, Dirty], [B, Clean], [B, Dirty]

      \part \

        \begin{subparts}
            \subpart After three time steps, we will have percieved 3 sets of percepts (6 percepts in total).
            In the next time period we will again have 4 sets of precepts: \\ $[A, clean], [B, clean], [A, dirty], [B, dirty]$
            \\

            \subpart After 3 time steps, we can have $4^{3} = 64$ possible sequences of these pairings
        \end{subparts}

      \part Storing all possible precepts means at each timepoint we store $[Position, Cleanliness]$.
        As we have seen, this pairing can take four values. \\
        If we store the all the possible sequences at each possible time point t, then at each timepoint we have $4^{t}$ entries.
        Summing across all time points, we get the following equation:
        \begin{equation*}
            \begin{gathered}
                4^T
            \end{gathered}
        \end{equation*}

    \end{parts}

  \vspace{1em}

  \question
    This is a rational agent, because it takes the best possible action based on the performance measure.
    It cleans the square if it is dirty, and moves to the other square if the current square is not dirty.
    It can not obtain a more optimal solution with the information it has.

  \vspace{1em}

  \question

    \begin{parts}

      \part A, C, G, H, F, I
      \part B, D, E
      \part A, C, H
      \part B, E, F, H, I
    \end{parts}

  \vspace{1em}

  \question

    \begin{parts}
      \part \

      \includeimage{figure_5a}

      The solution path Uniform Cost Search takes is A C H I, totalling a cost of 8.
      Uniform-cost Search is guaranteed to find the optimal solution. It searches all possible paths that have a cost
      less than the lowest-cost discovered goal state. By searching all lower-costing paths, it will either find a
      cheaper goal state, or discover that the selected goal state is the cheapest path available.

      \pagebreak

      \part \

      \includeimage{figure_5b}

      The solution path Iterative Deepening Search takes is A C G I, totalling a cost of 10. We know that this is not
      guaranteed to be optimal, because we have already seen a more optimal solution using Uniform Cost Search.
      Iterative Deepening Search does not take into account the cost when making it's decisions, but rather searches the
      possible paths until it finds any solution.

    \end{parts}

    \end{questions}

\end{document}
