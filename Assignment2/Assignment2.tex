% !TEX program = xelatex
\documentclass{exam}

\usepackage{amsmath}
\usepackage{amssymb}
\usepackage{graphicx}

\graphicspath{ {./images/} }

\renewcommand{\thepartno}{\roman{partno}}
\renewcommand{\thesubpart}{(\roman{subpart})}
\renewcommand{\subpartlabel}{\thesubpart} % removes dot
\qformat{Exercise \thequestion: \hfill}

\newcommand{\tab}{\hspace{.5cm}}
\newcommand{\includeimage}{\noindent\includegraphics[width=\linewidth]}


\title{Artificial Intelligence \\ Assignment 2 \\ Tutorial 4}
\author{Sudharshini Thangamurugan, 2577870 \\ Jeffrey Solomon, 2577386 \\ Ilya Senatorov, 2576798}
\date{May 7, 2019}

% \includeimage{FIGURE}

\begin{document}

  \maketitle

  \begin{questions}
    \setcounter{question}{5}
    \question
      \begin{parts}
        \part \

        \includeimage{figure6_a}

        Solution path = A C H I \\
        This solution is optimal because $h$ is consistent.
        \pagebreak

        \part \

        \includeimage{figure6_b}

        This algorithm found the solution at node I, via the path A C G I.
        This solution is not optimal.

        \part
        Yes, the hill-climbing algorithm can stop in a local minimum without finding a solution.
        If we use the cost as our heuristic, it will follow all the minimum cost nodes first. This
        will cause it to step into and get stuck in D, before we have tried any of the paths along C.
      \end{parts}
    \vspace{1em}

    \question

    % Admissable Heuristics, 2.5 points

    \question

    \begin{parts}
      \part
        \begin{align*}
          & (P \to R) \to (Q \leftrightarrow ~R) \\
          & \equiv \neg (\neg P \lor R) \lor ((Q \to \neg R) \land (\neg R \to Q)) \\
          & \equiv (P \land R) \lor ((\neg Q \lor \neg R) \land (\neg \neg R \lor Q)) \\
          & \equiv (P \land R) \lor ((\neg Q \lor \neg R) \land (R \lor Q)) \\
          & \equiv (P \land R) \lor ((\neg Q \land R) \lor (Q \land \neg R) \lor (Q \land \neg Q) \lor (R \land \neg R)) \\
          & \equiv (P \land R) \lor ((\neg Q \land R) \lor (Q \land \neg R)) \\
        \end{align*}

      \part
        \begin{align*}
          & \neg P \land (Q \lor \neg (\neg R \lor Q)) \\
          & \equiv \neg P \land (Q \lor (R \land Q)) \\
          & \equiv \neg P \land Q \\
        \end{align*}

      \part
        \begin{align*}
          & \neg (A \to \neg (\neg B \lor \neg C))\\
          & \equiv \neg (A \to (B \land C))\\
          & \equiv \neg (\neg A \lor (B \land C))\\
          & \equiv A \land \neg (B \land C))\\
          & \equiv A \land (\neg B \lor \neg C))\\
          & \equiv (A \land \neg B) \lor (A \land \neg C))\\
        \end{align*}

      \part
        \begin{align*}
          & (A \leftrightarrow \neg D) \lor (\neg C \land B)\\
          & \equiv ((A \to \neg D) \land (\neg D \to A)) \lor (\neg C \land B)\\
          & \equiv ((\neg A \lor \neg D) \land (D \lor A)) \lor (\neg C \land B)\\
          & \equiv ((A \land \neg D) \lor (\neg A \land D) \lor (A \land \neg A) \lor (D \land \neg D)) \lor (\neg C \land B) \\
          & \equiv (A \land \neg D) \lor (\neg A \land D) \lor (\neg C \land B) \\
        \end{align*}
    \end{parts}

  \end{questions}

\end{document}
