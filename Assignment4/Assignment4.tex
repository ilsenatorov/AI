% !TEX program = xelatex
\documentclass[11pt, letterpaper]{exam}

\usepackage{amsmath}
\usepackage{amssymb}
\usepackage{graphicx}

\graphicspath{ {./images/} }

\renewcommand{\thesubpart}{(\roman{subpart})}
\renewcommand{\subpartlabel}{\thesubpart} % removes dot
\qformat{Exercise \thequestion: \hfill}

\newcommand{\tab}{\hspace*{.5cm}}
\newcommand{\includeimage}{\noindent\includegraphics[width=\linewidth]}


\title{Artificial Intelligence \\ Assignment  4 \\ Tutorial 4: Thorsten Klößner}
\author{Sudharshini Thangamurugan, 2577870 \\ Jeffrey Solomon, 2577386 \\ Ilya Senatorov, 2576798}
\date{May 21, 2019}

% \includeimage{FIGURE}

\begin{document}

  \maketitle

  \begin{questions}
    \setcounter{question}{13}
    \question
    A formula is valid if all possible values return true.
    In CNF form, a clause is made up of items connected with '$\lor$'. $P \lor \neg P$ is a tautology, which by definition is valid.
    If we have just one or the other, but not both, we can always have a false value.
    $P$ will always be false by being 0, and $\neg P$ will always be false when $P = 1$.
    Unless we have both $P \lor \neg P$, any clause can be false. If any clause in CNF is false,
    then the whole thing is false, because CNF by definition is clauses connected by '$\land$'.
    Thus, each clause needs any item and its negation to be present in order to be valid.

    \vspace{1em}
    \question

    \begin{parts}
      \part \
        \begin{subparts}
          \subpart $\forall_x\exists_y[Student(x) \land Course(y) \land Takes(x, y)
          \to \exists_z((Student(z) \land \neg Likes(z, y)))]$

          \subpart $\forall_x \forall_y \forall_z[Student(x) \land Course(y) \land Takes(x, y) \land Course(z)
          \to Equals(y, z) \land \neg Likes(z)]$
        \end{subparts}

      \part \
        \begin{subparts}
          \subpart In every family there is a member who likes Elvis.
          \subpart Every boy who owns a red car has a cool grandfather or is rich.
        \end{subparts}

    \end{parts}

    \vspace{1em}
    \question

    \begin{parts}
      \part
      \begin{align*}
        step1\tab\varphi_1 & = \exists x [A(x) \to \forall x \forall y [B(x) \land C(x,y)]] \\
        step2\tab\varphi_1 & = \exists x [\neg A(x) \lor \forall x \forall y [B(x) \land C(x,y)]] \\
        step3,4\tab\varphi_1 & = \exists x \forall a \forall b [\neg A(x) \lor [B(a) \land C(a,b)]] \\
        step5\tab\varphi_1 & = \forall a \forall b [\neg A(f) \lor [B(a) \land C(a,b)]] \\
        step6\tab\varphi_1 & = \forall a \forall b [(\neg A(f) \lor B(a) \land (\neg A(f) \lor C(a,b))] \\
        step7\tab\varphi_1 & = \{\{\neg A(f) , B(a) \}, \{ \neg A(f), C(a,b)\}\} \\
        step8\tab\varphi_1 & = \{\{\neg A(f) , B(a) \}, \{ \neg A(g), C(c,d)\}\} \\
      \end{align*}
      \part
      \begin{align*}
        step1\tab \varphi_2 & = \forall x \exists y [[A(x,y) \land B(y)] \to \forall z (C(y,z) \lor D(x)) ] \\
        step2\tab\varphi_2 & = \forall x \exists y [ \neg [A(x,y) \land B(y)] \lor \forall z (C(y,z) \lor D(x))] \\
        step3\tab\varphi_2 & = \forall x \exists y [[ \neg A(x,y) \lor \neg B(y)] \lor \forall z (C(y,z) \lor D(x))] \\
        step5\tab\varphi_2 & = \forall x \exists y \forall z [ \neg A(x,y) \lor \neg B(y) \lor  C(y,z) \lor D(x)] \\
        step6\tab\varphi_2 & = \forall x \forall z [ \neg A(x,f(x)) \lor \neg B(f(x)) \lor  C(f(x),z) \lor D(x)] \\
        step7\tab\varphi_2 & = \{\{ \neg A(x,f(x)) , \neg B(f(x)) ,  C(f(x),z) , D(x) \}\}
      \end{align*}
      \part
      \begin{align*}
        step1\tab\varphi_3 & = \forall x \forall y \neg \forall z [A(x,y) \leftrightarrow (B(y) \land D(y,z)]  \\
        step1\tab\varphi_3 & = \forall x \forall y \neg \forall z [(A(x,y) \to (B(y) \land D(y,z)) \land ((B(y) \land D(y,z)) \to A(x,y))]  \\
        step2\tab\varphi_3 & = \forall x \forall y \neg \forall z [(A(x,y) \lor \neg (B(y) \land D(y,z)) \land (\neg A(x,y) \lor (B(y) \land D(y,z))] \\
        step2\tab\varphi_3 & = \forall x \forall y \forall z [(A(x,y) \lor \neg (B(y) \land \neg D(y,z)) \land (\neg A(x,y) \lor (B(y) \land  D(y,z))] \\
        step2\tab\varphi_3 & = \forall x \forall y \exists z [\neg[(A(x,y) \lor \neg (B(y) \land \neg D(y,z)) \land (\neg A(x,y) \lor (B(y) \land  D(y,z))]] \\
        step2\tab\varphi_3 & = \forall x \forall y \exists z [\neg(A(x,y) \lor \neg (B(y) \land \neg D(y,z)) \lor \neg (\neg A(x,y) \lor (B(y) \land  D(y,z))] \\
        step2\tab\varphi_3 & = \forall x \forall y \exists z [(\neg A(x,y) \land (B(y) \land \neg D(y,z)) \lor ( A(x,y) \land \neg (B(y) \land  D(y,z))] \\
        expand\tab\varphi_3 & = \forall x \forall y \exists z [(\neg A(x,y) \lor A(x,y)) \land (\neg A(x,y) \lor \neg B(y) \lor \neg D(y,z)) \\
        \land & (A(x,y) \lor (B(y) \land  D(y,z))) \land  ((B(y) \land  D(y,z)) \lor \neg (B(y) \land  D(y,z)))] \\
        simplify\tab\varphi_3 & = \forall x \forall y \exists z [(\neg A(x,y) \lor \neg B(y) \lor \neg D(y,z)) \land (A(x,y) \lor (B(y) \land  D(y,z))] \\
        step5\tab\varphi_3 & = \forall x \forall y \exists z [(\neg A(x,y) \lor \neg B(y) \lor \neg D(y,z)) \land (A(x,y) \lor B(y)) \land (A(x,y) \lor D(y,z))] \\
        step6\tab\varphi_3 & = \forall x \forall y  [(\neg A(x,y) \lor \neg B(y) \lor \neg D(y,f(x,y))) \land (A(x,y) \lor B(y)) \land (A(x,y) \lor D(y,f(x,y)))] \\
        step7\tab\varphi_3 & = \{\{ (\neg A(x,y) , \neg B(y) , \neg D(y,f(x,y))  \}, \{ A(x,y) , B(y)\}, \{ A(x,y) , D(y,f(x,y)) \}\} \\
        step8\tab\varphi_3 & = \{\{ (\neg A(x,y) , \neg B(y) , \neg D(y,f(x,y))  \}, \{ A(a,b) , B(b)\}, \{ A(c,d) , D(d,f(c,d)) \}\}
      \end{align*}

    \end{parts}

    \vspace{1em}

    \question
    \begin{align*}
      &\forall_x[(Child(x) \lor SweetTooth(x)) \to LikesCandy(x)] \\
      &\forall_x[(\neg Child(x) \land \neg SweetTooth(x)) \lor LikesCandy(x)] \\
      &\forall_x[(\neg Child(x) \lor LikesCandy(x)) \land (\neg SweetTooth(x)) \lor LikesCandy(x))] \\\\
      &HU(\theta \*) = \{Max, Anna\} \\
      &HE(\theta \*) = \{[\neg Child(Max) \lor LikesCandy(Max)], [\neg SweetTooth(Max) \lor LikesCandy(Max)], \\
      &HE(\theta \*) = \{[\neg Child(Anna) \lor LikesCandy(Anna)], [\neg SweetTooth(Anna) \lor LikesCandy(Anna)], \\
      & Child(Max), \neg LikesCandy(Max), SweetTooth(Anna), \neg LikesCandy(Anna)
      \}
    \end{align*}

    As we can see from the Skolem Normal Form, Max must either like candy or not be a child. Since we know he is a child,
    he therefore must like candy. Similarly with Anna, she must either like candy or not have a sweet tooth. Since we know
    she has a sweet tooth, she must like candy.


  \end{questions}

\end{document}
